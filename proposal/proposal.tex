\documentclass{scrartcl}

\usepackage[hidelinks]{hyperref}
\usepackage[none]{hyphenat}

\title{Essay Proposal}
\subtitle{COMP230-Ethics and Professionalis}

\author{Tristan Barlow-Griffin}

\begin{document}

\maketitle

\section*{My essay will be on: What are the negative effects industry professionals can experience during "Crunch Time"?}

Within the games industry "Crunch-Time" is a well established time in the development cycle of a product. I have found, through personal interactions with ex-industry professionals, that Crunch-Time is often part of the cycle where productivity is highest. In this paper, the author aims to highlight the consequences of Crunch Time on the employees and the final product. The introduction will briefly mention all of the problems Crunch-Time can produce and, provide the essay motivation through opinion pieces, written by industry professionals about Crunch-Time. I will discuss the damage that this kind of work style can have on the moral of the employees. The discussion will also include how the final may be impacted and how this can affect the longevity of the companies involved. By using studies from other industries I will form an argument against this type of work structure, suggesting working methods that remove the necessity that some companies think crunch is. To conclude I will weigh the merits of the strategies currently employed to remove the crunch period and explore methods that can make reduce the impact if a company does have to go into Crunch.
% Add details as appropriate.

\section*{Paper 1}
% This is an example! Replace the details with a paper relevant to your chosen topic.
\begin{description}
\item[Title:]Crunch Time: The Reasons and Effects of Unpaid Overtime in the Games Industry
\item[Citation:] \cite{edholm2017crunch}
\item[Abstract:] The games industry is notorious for its intense work ethics with uncompensated overtime and weekends at the office, also known as crunch or crunch time. Since crunch time is so common within the industry, is it possible that the benefits of crunch time outweigh the disadvantages? By studying postmortems and conducting interviews with employees in the industry, we aim to characterise crunch time and discover its effects on the industry. We provide a classification of crunch, i.e., four types of crunch which all have distinct characteristics and affect the product, employees and schedule in various ways. One of the crunch types stands out from the others by only having positive effects on product and schedule. A characteristic that all of the types have in common is an increase in stress levels amongst the employees. We identify a set of reasons for crunch and show that crunch is less pronounced in game studios where prioritisation of features is a regular practice.
\item[Web link:] \url{http://ieeexplore.ieee.org.ezproxy.falmouth.ac.uk/document/7965428/}
\item[Full text link:] \url{http://ieeexplore.ieee.org.ezproxy.falmouth.ac.uk/document/7965428/}
\item[Comments:] 
\end{description}

\section*{Paper 2}
\begin{description}
\item[Title: ]Overtime reduction, work-life balance, and psychological well-being for research and development engineers in Japan
\item[Citation:] \cite{fujimoto2013overtime}
\item[Abstract:] Due to the recent intensification of work hour regulation by Japanese employers, a large number of engineers in research and development are increasingly unable to put in satisfactory amount of time in work, and the incidences of experiencing non-accomplishment in work are growing. This study investigates whether and how the overtime reduction affects engineers' sense of fulfillment in work and personal life, and examines how they influence engineers' depression and perceived health. Results show that overtime reduction has both positive and negative sides, and while on one hand it enhances the time adequacy for engineers' private life, on the other hand it significantly reduces their sense of fulfillment in work. Furthermore, unbalance such as “non-achievement in work” and “fulfillment in private life” is likely to deteriorate their psychological well-being. Findings suggest that in order to realize engineers' work-life balance it is not sufficient to just enhance the time in private life, but it is also necessary to ensure simultaneously a sufficient time in work.
\item[Web link:]\url{http://ieeexplore.ieee.org.ezproxy.falmouth.ac.uk/document/6962662/}
\item[Full text link:]\url {http://ieeexplore.ieee.org.ezproxy.falmouth.ac.uk/stamp/stamp.jsp?arnumber=6962662}
\item[Comments:]
\end{description}

\section*{Paper 3}
\begin{description}
\item[Title:]Not going to take this anymore: multi-objective overtime planning for software engineering projects
\item[Citation:] \cite{ferrucci2013not}
\item[Abstract:] Software Engineering and development is well-known to suffer from unplanned overtime, which causes stress and illness in engineers and can lead to poor quality software with higher defects. In this paper, we introduce a multi-objective decision support approach to help balance project risks and duration against overtime, so that software engineers can better plan overtime. We evaluate our approach on 6 real world software projects, drawn from 3 organisations using 3 standard evaluation measures and 3 different approaches to risk assessment. Our results show that our approach was significantly better (p <; 0.05) than standard multi-objective search in 76 percent of experiments (with high Cohen effect size in 85 percent of these) and was significantly better than currently used overtime planning strategies in 100 percent of experiments (with high effect size in all). We also show how our approach provides actionable overtime planning results and investigate the impact of the three different forms of risk assessment.
\item[Web link:] \url{ http://ieeexplore.ieee.org.ezproxy.falmouth.ac.uk/document/6606592/}
\item[Full text link:]\url{ http://ieeexplore.ieee.org.ezproxy.falmouth.ac.uk/stamp/stamp.jsp?arnumber=6606592}
\item[Comments:] 
\end{description}

\section*{Paper 4}
\begin{description}
\item[Title:] A simulation study to reduce nurse overtime and improve patient flow time at a hospital endoscopy unit
\item[Citation:] \cite{taheri2012simulation}
\item[Abstract:]Increasing demand for endoscopic procedures, coupled with decreasing insurance reimbursement, has necessitated improvement in endoscopy unit operational performance measures, such as increasing throughput and reducing staff overtime without an increase in patient waiting time. In pursuit of improving these measurements, maintaining the nurse-to-patient ratio requirements in the recovery area throughout the clinic's operation time is a challenging problem for endoscopy units. To maintain compliance with this ratio, patients occasionally have to be held in the procedure rooms during the clinic's peak time. On the other hand, level loading could potentially increase the amount of overtime. In this paper, we describe our efforts to use discrete event simulation to investigate the impact of several strategies to address the minimum recovery nurse requirements in the endoscopy unit of Duke University Medical Center. Our objective was to minimize patient flow times and nurse overtime while sustaining the required nurse-patient staffing ratio in recovery.
\item[Web link:]\url {http://ieeexplore.ieee.org.ezproxy.falmouth.ac.uk/document/6465136/}
\item[Full text link:]\url {http://ieeexplore.ieee.org.ezproxy.falmouth.ac.uk/stamp/stamp.jsp?arnumber=6465136}
\item[Comments:] 
\end{description}

\section*{Paper 5}
\begin{description}
\item[Title:]A case study on the impact of scrum on overtime and customer satisfaction
\item[Citation:] \cite{mann2005case}
\item[Abstract:]This paper provides results, and experiences from a longitudinal, 2 year industrial case study. The quantitative results indicate that after the introduction of a scrum process into an existing software development organization the amount of overtime decreased, allowing the developers to work at a more sustainable pace while at the same time the qualitative results indicate that there was an increase in customer satisfaction.
\item[Web link:]\url {http://ieeexplore.ieee.org.ezproxy.falmouth.ac.uk/document/1609806/}
\item[Full text link:] \url {http://ieeexplore.ieee.org.ezproxy.falmouth.ac.uk/stamp/stamp.jsp?arnumber=1609806}
\item[Comments:]
\end{description}

\section*{Paper 6}
\begin{description}
\item[Title:] Software development and crunch time; and more
\item[Citation:] \cite{onlineWebl}
\item[Abstract:]Ruben Ortega discusses developers and crunch time; Mark Guzdial considers the impact of open source practices on computing education; and Daniel Reed writes about the technological shift from computational paucity to computational plethora.
\item[Web link:] \url {https://dl-acm-org.ezproxy.falmouth.ac.uk/citation.cfm?id=1785419&CFID=993670308&CFTOKEN=23525870}
\item[Full text link:]\url { http://cacm.acm.org.ezproxy.falmouth.ac.uk/magazines/2010/7/95050/fulltext}
\item[Comments:] 
\end{description}

\section*{Paper 7}
\begin{description}
\item[Title:]Cutting Corners and Working Overtime: Quality Erosion in the Service Industry
\item[Citation:] \cite{oliva2001cutting}
\item[Abstract:] The erosion of service quality throughout the economy is a frequent concern in the popular press. The American Customer Satisfaction Index for services fell in 2000 to 69.4 percent, down 5 percentage points from 1994. We hypothesize that the characteristics of services--inseparability, intangibility, and labor intensity--interact with management practices to bias service providers toward reducing the level of service they deliver, often locking entire industries into a vicious cycle of eroding service standards. To explore this proposition we develop a formal model that integrates the structural elements of service delivery. We use econometric estimation, interviews, observations, and archival data to calibrate the model for a consumer-lending service center in a major bank in the United Kingdom. We find that temporary imbalances between service capacity and demand interact with decision rules for effort allocation, capacity management, overtime, and quality aspirations to yield permanent erosion of the service standards and loss of revenue. We explore policies to improve performance and implications for organizational design in the service sector.
\item[Web link:]\url  {https://dl-acm-org.ezproxy.falmouth.ac.uk/citation.cfm?id=970050&CFID=993670308&CFTOKEN=23525870}
\item[Full text link:]\url { http://dl.acm.org.ezproxy.falmouth.ac.uk/ft_gateway.cfm?id=367022&ftid=50938&dwn=1&CFID=727945130&CFTOKEN=59144450}
\item[Comments:] 
\end{description}

\section*{Paper 8}
\begin{description}
\item[Title:]What went wrong? A survey of problems in game development
\item[Citation:] \cite{petrillo2009went}
\item[Abstract:]Despite its growth and profitability, many reports about game projects show that their production is not a simple task, but one beset by common problems and still distant from having a healthy and synergetic work process. The goal of this article is to survey the problems in the development process of electronic games, which are mainly collected from game postmortems, by exploring their similarities and differences to well-known problems in traditional information systems
\item[Web link:]\url  {https://dl-acm-org.ezproxy.falmouth.ac.uk/citation.cfm?id=1486521}
\item[Full text link:] \url {https://dl-acm-org.ezproxy.falmouth.ac.uk/ft_gateway.cfm?id=1486521&ftid=612586&dwn=1&CFID=993670308&CFTOKEN=23525870}
\item[Comments:]
\end{description}

\section*{Paper 9}
\begin{description}
\item[Title:] Impact of overtime and stress on software quality
\item[Citation:] \cite{akula2008impact}
\item[Abstract:] Producing software requires an environment conducive to concentration and measured work. In this paper several projects are presented with varying degrees of overtime where the resultant quality of the applications in terms of defect counts varies directly. It is proposed that stress in the work environment driven by overtime adversely affects software quality.
\item[Web link:] \url {https://www.researchgate.net/profile/James_Cusick/publication/259781769_Impact_of_Overtime_and_Stress_on_Software_Quality/links/0f31752dd77ad07cc0000000.pdf}
\item[Full text link:]\url  {https://www.researchgate.net/profile/James_Cusick/publication/259781769_Impact_of_Overtime_and_Stress_on_Software_Quality/links/0f31752dd77ad07cc0000000.pdf}
\item[Comments:] 
\end{description}

\section*{Paper 10}
\begin{description}
\item[Title:]What went right and what went wrong: an analysis of 155 postmortems from game development
\item[Citation:] \cite{whatwentright2016t}
\item[Abstract:] In game development, software teams often conduct postmortems to reflect on what went well and what went wrong in a project. The postmortems are shared publicly on gaming sites or at developer conferences. In this paper, we present an analysis of 155 postmortems published on the gaming site Gamasutra.com. We identify characteristics of game development, link the characteristics to positive and negative experiences in the postmortems and distill a set of best practices and pitfalls for game development.
\item[Web link:]\url  {https://www.microsoft.com/en-us/research/publication/what-went-right-and-what-went-wrong-an-analysis-of-155-postmortems-from-game-development/}
\item[Full text link:]\url  {https://www.microsoft.com/en-us/research/wp-content/uploads/2016/06/washburn-icse-2016-2.pdf}
\item[Comments:] 
\end{description}

\section*{Paper 11}
\begin{description}
\item[Title:]Overtime effects on project team effectiveness
\item[Citation:] \cite{olson2011overtime}
\item[Abstract:] The fast-paced, competitive, and continually changing business environment is driven by information technology projects. The changing scope, requirements, and time pressure of projects requires overtime and long sprints. As commitments to an implementation date are established, the project team is left to absorb the additional time required to address project issues, new requirements and constraints. The extra load requires the project team to invest additional resource hours in order to achieve the target implementation date. In this paper, the impacts of overtime on the project team are considered as well as the implications on the project output. Recommendations are made on organizational improvements to structuring overtime more effectively, more effective team and personal coping mechanisms for dealing withovertime stress, in order to lessen the influence of overtime on the quality of the projectdeliverables.
\item[Web link:]\url  {http://www.academia.edu/download/41279167/OVERTIME_EFFECTS_ON_PROJECT_TEAM_EFFECTI20160116-9161-2b6mkm.pdf}
\item[Full text link:]\url  {http://www.academia.edu/download/41279167/OVERTIME_EFFECTS_ON_PROJECT_TEAM_EFFECTI20160116-9161-2b6mkm.pdf}
\item[Comments:] 
\end{description}

\section*{Paper 12}
\begin{description}
\item[Title:] Influence of Overtime Work, Sleep Duration, and Perceived Job Characteristics on the Physical and Mental Status of Software Engineers
\item[Citation:] \cite{nishikitani2005influence}
\item[Abstract:]
To investigate the impact of overtime work, sleep duration, and perceived job characteristics on physical and mental status, a cross-sectional study was conducted on 377 workers (average age; 28 years old) in an information-technology (IT) company, engaged in consultation, system integration solution, and data management relevant to IT system. The psychophysical outcomes of overtime work were assessed using the Hamilton Depression Scale (HDS), Profile of Mood Status (POMS), major physical symptoms, and overtime work data for the preceding three-months. Sleep duration was directly asked by a physician. A job strain index was defined as the ratio of job-demands to job-control scores evaluated using the Job Content Questionnaire (JCQ). In a univariate analysis, overtime work was significantly related with HDS scores, POMS anger-hostility scores, and the total physical symptom count in both sexes (all p<0.05), but not in multiple regression models, after controlling for sleep duration and the job strain index. Sleep duration was negatively related to the symptom count in men and to POMS tension-anxiety scores in women (both p<0.05); the job strain index was positively related to POMS anger-hostility scores in both sexes and to HDS scores and POMS tension-anxiety scores in men (all p<0.05). Although overtime work was associated with physical and mental complaints, sleep duration and the job strain index seemed to be better indicators for physical and mental distress in overloaded workers.
\item[Web link:]\url  {https://www.jstage.jst.go.jp/article/indhealth/43/4/43_4_623/_article}
\item[Full text link:]\url  {https://www.jstage.jst.go.jp/article/indhealth/43/4/43_4_623/_pdf}
\item[Comments:] 
\end{description}

\bibliographystyle{ieeetran}
\bibliography{initial_references}

\end{document}
