% Please do not change the document class
\documentclass{scrartcl}

% Please do not change these packages
\usepackage[hidelinks]{hyperref}
\usepackage[none]{hyphenat}
\usepackage{setspace}
\doublespace

% You may add additional packages here
\usepackage{amsmath}

% Please include a clear, concise, and descriptive title
\title{What are the negative effects of crunch time to the mental states of game developers, and how will this affect the final product? }

% Please do not change the subtitle
\subtitle{COMP230- Ethics and Professionalism}

% Please put your student number in the author field
\author{1607804}

\begin{document}

\maketitle

%abstract{Please include an abstract of at most 100 words (these do not count towards your word count).}

\section{Introduction}In the games industry, the term "Crunch" or "Crunch Time" is used to describe periods of extreme workload \cite[p. 468]{edholm2017crunch}. Opinion pieces written by developers, who have experienced Crunch, describe an environment that limits or removes activities that do not contribute to the game. These activities include but is not limited to "family and even food"\cite{schreier_2017}. In this essay, the author will explore how this environment will affect the mental states of the employees involved. By using research theory and case studies from the games industry, we will explore how poor mental health can change the design process. Also, we will use examples from other industries to further support our insights. The author aims to provide insight how Crunch may affect the mental state of the developers. The reader will get a better understanding of the consequences of Crunch, reducing the negative ramifications to the final product.

\section{Discussion}
The development cycle of a product can be a smooth process that does not require crunch. However,  over-scoping, unforeseen barriers or poor planning may cause the project to fall behind schedule. If the team wants to deal with the remaining bugs, late feature requests and last minute modifications they have to start putting in over-time, thus we get crunch\cite{onlineWebl}\cite{edholm2017crunch}. During a satisfaction survey in 2016, the IGDA reported that 65\%  of industry professionals said that their job involved Crunch with an extra 32\%  saying their jobs "did require periods of long hours" it "was just not called 'crunch'"\cite[p.20]{weststarlegault2016}. From this survey, it is clear that more than half of the industry professionals experience this kind of overtime. The first effect of these conditions we will discuss is mental fatigue.


We define mental fatigue as the difficulty in starting or sustaining voluntary activities\cite{chaudhuri2004fatigue}. A sustained psychological load can cause this kind of fatigue in otherwise healthy humans\cite {mizuno2011mental}. The author argues that all professions within the games industry if performed to industry standard, constitute a significant mental load. The IGDA reported that during crunch 35\% of employees did a 50-59 hour working week, a further 28\% reported 60-69 and a smaller 13\% reported 70 or more hours\cite[p.20]{weststarlegault2016}. According to the UK government, it is illegal to work more than 48 hours a week averaged over 17 weeks, with acceptions\cite{maximumweeklyhours}. Arguably, by working well over the suggested 48 hours per week limit, performing great cognitive loads as found in game development, will cause employees to get mental fatigue. Researchers found that mentally fatigued workers have trouble keeping their attention focused and that they're distracted easily\cite{bartlett1943ferrier}. This lack of concentration may lead to the employees making mistakes. Some researchers believe that it's during these extended periods where errors are most likely to occur\cite{olson2011overtime}. When mistakes are made, so close to the deadline for a product, it's not surprising to find mistakes shipped in the final product. If bugs are present in the last build, to paying customers, it would not be unexpected for the company to lose consumer confidence.  It's not uncommon for a bad launch in the game industry to ruin a studios reputation. 


If mental fatigue is a short-term result of crunch, "Burnout" is a long-term mental state. In this essay, we define Burnout as "a syndrome of psychological problems experienced as a result of chronic work stress"\cite{milfont2008burnout}. Those that experience burnout feel cognitive, emotional and physical exhaustion\cite{sonnentag1994stressor}. Definitions include withdrawal or less involvement in their given profession\cite{sonnentag1994stressor}. Research from 1989 found that stress scores for software engineers were higher than other proffessions\cite{kumash1989work}. The main work-related causes were the pace of work and the amount of overtime\cite{kumash1989work}. A study performed in 2012 found that 68\% of information technology professionals from Iran had a very stressful occupation\cite{bolhari2012occupational}. A further 5\% had an extremely stressful occupation\cite{bolhari2012occupational}. While not strictly software engineering, we believe that the related industry is relevant to give rough numbers on stress levels. As stated above burnout can include complete withdrawal or less involvement in the given occupation. The author argues that less participation in projects will negatively impact the final product. 
\bibliographystyle{ieeetran}
\bibliography{references}

\end{document}
